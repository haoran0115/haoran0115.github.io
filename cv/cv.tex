\documentclass[12pt,a4paper,sans]{moderncv}
\usepackage[utf8]{inputenc}
\usepackage[scale=0.92]{geometry}
\usepackage{makecell}
\usepackage{booktabs}
\usepackage{soul}
\usepackage[version=3]{mhchem}
\usepackage{fontspec}
\usepackage{url}
\usepackage{lipsum}
% main font
% \usepackage{libertinus}
% \usepackage[amsthm]{libertinust1math}
\usepackage{lmodern}
% \usefonttheme{serif}
\usepackage{inconsolata}
\setmonofont{inconsolata}

\newfontfamily\libert{Linux Libertine}

% itemize
\usepackage{enumitem}
\setlist{nosep}

% \ul underline settings
\setul{1pt}{}

% tabular settings
\setlength{\tabcolsep}{5pt}

% % calibre font
% \usepackage{fontspec}
% \setmainfont{Calibri}

% % Chinese support
% \usepackage{xeCJK}
% \setCJKmonofont{SimSun}

% theme
\moderncvtheme[blue]{classic}

\recomputelengths
\firstname{\rmfamily \libert Haoran}
\familyname{\rmfamily \libert SUN}
% \firstname{Haoran}
% \familyname{SUN}
% \title{Undergraduate student}
\address{2001 Longxiang Road}{Longgang District}{Shenzhen, China}
\mobile{+86 139 1029 0104}
% \phone{+86 139 1029 0104}
\email{haoransun@link.cuhk.edu.cn}
\social[github]{haoran0115}
\homepage{haoran0115.github.io}
% \photo[64pt]{jdoe_picture}
% \quote{The world opens itself before those with noble hearts.}

% set width of cventry
\settowidth{\hintscolumnwidth}{Sep. 2021--Dec. 2021\ }

% re-define section as cvsection
% \newcommand{\cvsection}[1]{\section{\textsc{\rmfamily\libert #1}}}
\newcommand{\cvsection}[1]{\section{{#1}}}
\newcommand{\ntf}{\hl{\textbf{xxx}}}

% make content compact
\linespread{0.9}

\begin{document}
\maketitle
% \section{Basic information}


\cvsection{Education}
\cventry{Sep. 2019--Present}
    {B.Sc.}
    {}
    {Bioinformacis, Chinese University of Hong Kong, Shenzhen (CUHK-Shenzhen)}
    {}
    {\begin{tabular}{@{}lrr@{}}
        \textbf{Cumulative GPA} & 3.671/4.000 & rank 3/38\\
        \textbf{Major GPA} & 3.800/4.000 & rank 1/38\\
    \end{tabular}}
    % {Major GPA: 3.800/4.000, rank 1/38; cumulative GPA: 3.671/4.000, rank 3/38}


% \cvline{}{\footnotesize
%     \begin{tabular*}{0.78\textwidth}{@{\extracolsep{\fill}}cccc}
%         \toprule
%         Calculus I & Calculus II & \makecell[c]{General\\Chemistry} & \makecell[c]{Programming\\Methodology} \\ \midrule
%         3.70/4.00 & 4.00/4.00 & 4.00/4.00 & 4.00/4.00 \\ 
%         \toprule\\
%         \toprule
%         \makecell[c]{Computational\\Biology} & Biochem. & \makecell[c]{Linear\\Algebra} & \makecell[c]{Prob. and\\Stats. Inference} \\ \midrule
%         4.00/4.00 & 3.70/4.00 & 4.00/4.00 & 4.00/4.00 \\ \bottomrule
%     \end{tabular*}
% }


% \cvsection{Projects}
% \cvline{Nov. 2021}{Parallel tempering of 1-D particle using Fortran}

\cvsection{Research Experiences}
\cventry{Apr. 2021--Present}
    {Research assistant}{}
    {Hajime Hirao's group}{CUHK-Shenzhen}
    {}
\cvline{Apr. 2021--June 2021}{\small
    \textbf{Training:} theoretical learning of quantum chemistry
    \begin{itemize}[itemsep=4pt]
        \item Learning \textit{Modern Quantum Chemistry}
        \item SCF algorithm coding by Fortran
        \begin{itemize}
            \item RHF 6-31G \cee{H2} molecule
            \item UHF 6-31G \cee{H_2^-} molecule
        \end{itemize}
        \item SCF acceleration by DIIS algorithm
        \begin{itemize}
            \item \ul{Fixed problematic DIIS algorithm} in original group Fortran code
        \end{itemize}
    \end{itemize}
}
\cvline{June 2021--July 2021}{\small
    \textbf{Training:} reaction pathway analysis--hydroxylation reaction between P450 Cpd I and propane
    \begin{itemize}[itemsep=4pt]
        \item Scan along the reaction path
        \item Geometry optimization of intermediates
        \item Calculation under different spin states
        \item Writing scripts to extract information and generate report efficiently
    \end{itemize}
}
\cvline{Aug. 2021--Dec. 2021}{\small
    \textbf{Project:} reaction pathway analysis--P450 C-S bond formation by TleB (PDB ID: 6J83)
    \begin{itemize}[itemsep=4pt]
        % \setlength\itemsep{0.25em}
        \item \ul{Design the whole research plan}
        \item DFT calculation
        \begin{itemize}
            \item Build truncated model
            \item DFT calculations along the proposed reaction pathway
            \item Identify electronic configurations under different spin states
            \item Swap electrons in \textsf{\textit{α}} or \textsf{\textit{β}} orbitals to get stable configuration
        \end{itemize}
        \item Molecular dynamics simulation of initial reaction complex
        \begin{itemize}
            \item Calculate MM parameters, correct protonated state, setup model
            \item Perform MD simulations, check non-bonding interactions, check clusters in trajectory by
            statistical algorithms
            \item MMPBSA free energy approximation to compare population between states
        \end{itemize}
        \item QM/MM
        \begin{itemize}
            \item Determine QM region of the system
            \item Use MM parameters to build up QM/MM model
            \item Use small basis set when performing optimization, then use large basis set and
            electronic embedding scheme to investigate electronic configurations and effect of
            protein
        \end{itemize}
    \end{itemize}
}
\cvline{Apr. 2022--Present}{\small
    \textbf{Project:} EDA and NBO analysis of the nature of coordinate bond at the
    heme iron center in cytochrome P450 inhibition
    \begin{itemize}[itemsep=4pt]
        \item Write an example Lewis configuration for NBO input
        \item EDA analysis using Q-Chem
        \begin{itemize}
            \item Fix convergence problem by shutdown DIIS when error is small
        \end{itemize}
    \end{itemize}
}

\cventry{Jan. 2020--Dec. 2020}
{Research assistant}{}
{Hsien-da Huang's group}{CUHK-Shenzhen}
{}
\cvline{}{\small
    \textbf{Project:} effects of traditional Chinese medicine in gene regulation:
    identify DEGs using statistical methods
    \begin{itemize}[itemsep=4pt]
        \item Visualization of gene expression profile using PCA and t-SNE
        \item Group tutorial about how to use Connectivity Map
        \begin{itemize}
            \item Exploring databases, submitting a query, interpreting statistics and heatmap
        \end{itemize}
        \item Gene set enrichment analysis (GSEA) for traditional Chinese medicines perturbed
        gene expression profile to identify differentially expressed gene sets
    \end{itemize}
}

\cvsection{Skills}
% \cvline{}{Please refer to Github repositories for reference: \href{www.github.com/haoran0115}{haoran0115}}
\cvline{Coding langs}{Python, Fortran, CUDA C++ and CUDA Fortran (elementary), MATLAB, \LaTeX}
\cvline{Computer skills}{Linux (including system configuration, multi-user management, software
compliation and installation), WSL, Git}
% \cvline{Scientific computing}{Numpy, MATLAB, MKL (elementary)}
\cvline{Programming tools}{Vim, VSCode, Jupyter Lab, Windows Terminal}
\cvline{Compt. chem. tools}{Amber, Gromacs, Q-Chem, Gaussian, VMD, Autodock Tools}


\cvsection{Teaching Experiences}
\cventry{Sep. 2021--Dec. 2021}
{Undergraduate student teaching fellow (USTF)}{}
{Computational Biology (BIM2005)}{CUHK-Shenzhen}
{
    \begin{itemize}[itemsep=4pt]
        \item Create a slide about how to simplify the Schrödinger equation of hydrogen atom using
        atomic units
        \item Tutorial session: molecular docking tool Autodock-Vina
        \item Tutorial session: review basic principles of quantum mechanics and quantum chemistry
        \item Tutorial session: mathematical background and hands-on Python implementation of principal component decomposition (PCA) algorithm
        \item Hold office hours, homework grading, exam invigilation
    \end{itemize}
}

        
\vspace{0.5em}
\cventry{Jan. 2022--May 2022\newline}
{Undergraduate student teaching fellow (USTF)}{}
{Organic Chemistry (BIO2003)}{CUHK-Shenzhen}
{
    \begin{itemize}[itemsep=4pt]
        \item Tutorial session: basic concepts and exercises of stereochemistry
        \item Tutorial session: detailed mechanism of keto-enol tautomerism, aldol reaction,
        and Claisen condensation reaction, related exercises
        \item Hold office hours, homework grading, exam invigilation
    \end{itemize}
}


% \cvsection{Research Interests}
% \cvline{}{Computational chemistry/biology,
% development of free-energy calculation and TS-searching algorithms,
% statistical learning applied to computational chemistry/biology.}

\cvsection{Achievements and Honors}
\cventry{Sep. 2018}
{The First prize}{}
{Chinese Chemistry Olympiad, provincial level}{}
{}
\cventry{Sep. 2019--June 2023\newline (expected)}
{Bowen Scholarship}{}
{30,000 RMB/year (total 120,000 RMB), CUHK-Shenzhen}{}
{}
\cventry{Sep. 2020}
{Academic Year 2019-20 Dean's List Award}{}
{School of Science and Engineering, CUHK-Shenzhen}{}
{}
\cventry{Sep. 2021}
{Academic Year 2020-21 Dean's List Award}{}
{School of Life and Health Sciences, CUHK-Shenzhen}{}
{}
\cventry{Sep. 2021}
{The Second prize}{}
{Contemporary Undergraduate Mathematical Contest in Modeling, provincial level}{}
{}

\cvsection{Courses Taken}
\cvline{Math \& stat}{
    Calculus I and II, ordinary differential equations, linear algebra, advanced linear algebra,
    probability and statistics I
}
\cvline{Chem \& physics}{
    Mechanics, organic chemistry, physical chemistry I,
    computational (structural) biology, computational biology laboratory, biophysics,
    molecular simulation \& modeling I
    (including statistical mechanics theories, monte carlo, MD simulation algorithms)
}
\cvline{Informatics}{
    Introduction to computer science: programming methodology,
    computational laboratory,
    bioinformacis,
    computational genomics and proteomics,
    machine learning in computational biology,
    design and analysis of bioinformacis algorithms
}
\cvline{Biology}{
    Biochemistry, cell and molecular biology, genetics
}

\cvsection{Language Skills}
\cvline{}{Chinese (native)\newline
    English (GRE Q155) \newline
    Japanese (elementary, only able to read)
}


% \section{Hobby}
% \cvline{hobby 1}{\small Description}


\end{document}


