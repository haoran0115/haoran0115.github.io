\documentclass[12pt,a4paper,sans]{moderncv}
% \usepackage[utf8]{inputenc}
\usepackage{fontspec}
\usepackage[scale=0.92,top=1cm]{geometry}
\usepackage{makecell}
\usepackage{booktabs}
\usepackage{soul}
\usepackage[version=3]{mhchem}
\usepackage{fontspec}
\usepackage{url}
\usepackage{lipsum}
% main font
% \usepackage{libertinus}
% \usepackage[amsthm]{libertinust1math}
\usepackage{lmodern}
% \usefonttheme{serif}
\usepackage{inconsolata}
\setmonofont{inconsolata}

\newfontfamily\libert{Linux Libertine}

% itemize
\usepackage{enumitem}
\setlist{nosep}

% \ul underline settings
\setul{1pt}{}

% tabular settings
\setlength{\tabcolsep}{5pt}

% external link icon/symbol 
\usepackage{fontawesome5}
\newcommand\linksymbol{\scriptsize\faExternalLink*}

% % calibre font
% \usepackage{fontspec}
% \setmainfont{Calibri}

% % Chinese support
% \usepackage{xeCJK}
% \setCJKmonofont{SimSun}

% long cventry
% A custom version of the \cventry command that supports large itemized lists
% inside argument #7 (the custom cvitemize lists should be used!)
\newcommand*{\cventrylong}[7][.25em]{%
  \begin{tabular}{@{}p{\hintscolumnwidth}@{\hspace{\separatorcolumnwidth}}p{\maincolumnwidth}@{}}%
    \raggedleft\hintstyle{#2} &{%
        {\bfseries#3}%
        \ifthenelse{\equal{#4}{}}{}{, {\slshape#4}}%
        \ifthenelse{\equal{#5}{}}{}{, #5}%
        \ifthenelse{\equal{#6}{}}{}{, #6}%
    }%
  \end{tabular}%
  {\small#7}%
  \par\addvspace{#1}}
% A custom version of the itemize environment that sets the appropriate left
% margin for use inside \cventylong
\newlist{cvitemize}{itemize}{1}
\setlist[cvitemize]{label=\labelitemi,%
leftmargin=\hintscolumnwidth+\separatorcolumnwidth+\labelwidth+\labelsep}

% theme
\moderncvtheme[blue]{classic}

\recomputelengths
\firstname{\rmfamily \libert Haoran}
\familyname{\rmfamily \libert SUN}
% \firstname{Haoran}
% \familyname{SUN}
% \title{Undergraduate student}
\address{2001 Longxiang Road}{Longgang District}{Shenzhen, China}
\mobile{+86 139 1029 0104}
% \phone{+86 139 1029 0104}
\email{haoransun@link.cuhk.edu.cn}
\social[github]{haoran0115}
\homepage{haoran0115.github.io}
% \photo[64pt]{jdoe_picture}
% \quote{The world opens itself before those with noble hearts.}

% set width of cventry
\settowidth{\hintscolumnwidth}{Sep. 2021--Dec. 2021\ }

% re-define section as cvsection
% \newcommand{\cvsection}[1]{\section{\textsc{\rmfamily\libert #1}}}
\newcommand{\cvsection}[1]{\section{{#1}}}
\newcommand{\ntf}{\hl{\textbf{xxx}}}

% make content compact
\linespread{0.9}

% fix url problem
\AtBeginDocument{\hypersetup{baseurl={}}}

\begin{document}
\maketitle
% \section{Basic information}

\vspace{-1em}
\cvsection{Education}
\cventry{Sep. 2019--Present}
    {B.Sc.}
    {}
    {Bioinformacis, Chinese University of Hong Kong, Shenzhen (CUHK-Shenzhen)}
    {}
    {\begin{tabular}{@{}lrlr@{}}
        \textbf{GPA, cumulative} & 3.716/4.000 & \textbf{rank} & 1/39\\
        \textbf{GPA, major} & 3.831/4.000      & \textbf{rank} & 1/39
    \end{tabular}}
\cventry{June 2022--Aug. 2022}
    {Summer visiting program}
    {}
    {University of California, Berkeley (UCB)}
    {}
    {Courses taken: MATH104 introduction to real analysis,
    MATH128A numerical analysis, CS61C machine structure\\
    \begin{tabular}{@{}lrr@{}}
        \textbf{GPA} & 4.000/4.000 &  \\
    \end{tabular}}


% \cvline{}{\footnotesize
%     \begin{tabular*}{0.78\textwidth}{@{\extracolsep{\fill}}cccc}
%         \toprule
%         Calculus I & Calculus II & \makecell[c]{General\\Chemistry} & \makecell[c]{Programming\\Methodology} \\ \midrule
%         3.70/4.00 & 4.00/4.00 & 4.00/4.00 & 4.00/4.00 \\ 
%         \toprule\\
%         \toprule
%         \makecell[c]{Computational\\Biology} & Biochem. & \makecell[c]{Linear\\Algebra} & \makecell[c]{Prob. and\\Stats. Inference} \\ \midrule
%         4.00/4.00 & 3.70/4.00 & 4.00/4.00 & 4.00/4.00 \\ \bottomrule
%     \end{tabular*}
% }


% \cvsection{Projects}
% \cvline{Nov. 2021}{Parallel tempering of 1-D particle using Fortran}

\cvsection{Skills}
% \cvline{}{Please refer to Github repositories for reference: \href{www.github.com/haoran0115}{haoran0115}}
% \cvline{}{\textit{Remark:} Github repositories may be a reference for those skills}
\cvline{Coding langs}{Python, Fortran, C, CUDA C++ and CUDA Fortran (elementary), MATLAB, \LaTeX}
\cvline{Computer skills}{Linux (including system configuration, multi-user management, software
compliation and installation), WSL, Git}
% \cvline{Scientific computing}{Numpy, MATLAB, MKL (elementary)}
\cvline{Programming tools}{Vim, VSCode, Jupyter Lab, Windows Terminal}
\cvline{Scientific softs}{Amber, Gromacs, Q-Chem, Gaussian, VMD, Autodock Tools}


\cvsection{Teaching Experiences}
\cventry{Sep. 2021--Dec. 2021}
{Undergraduate student teaching fellow}{}
{computational biology}{CUHK-Shenzhen}
{
    \begin{itemize}[itemsep=2pt]
        \item Create a slide about how to simplify the Schrödinger equation of hydrogen atom using
        atomic units
        \item Tutorial sessions: molecular docking tool Autodock-Vina;
        review basic principles of quantum mechanics and quantum chemistry;
        mathematical background and hands-on Python implementation of principal component decomposition (PCA) algorithm
        \href{https://github.com/haoran0115/pca-implementation}{\linksymbol}
        \item Hold office hours, homework grading, exam invigilation
    \end{itemize}
}
\vspace{0.5em}
\cventry{Jan. 2022--May 2022\newline}
{Undergraduate student teaching fellow}{}
{organic chemistry}{CUHK-Shenzhen}
{
    \begin{itemize}[itemsep=2pt]
        \item Tutorial sessions: basic concepts and exercises of stereochemistry;
        detailed mechanism of keto-enol tautomerism, aldol reaction,
        and Claisen condensation reaction, related exercises
        \item Hold office hours, homework grading, exam invigilation
    \end{itemize}
}
\
\cvsection{Achievements and Honors}
\cventry{Sep. 2018}
{The First prize}{}
{Chinese Chemistry Olympiad}{}
{}
\cventry{Sep. 2019--June 2023}
{Bowen Scholarship}{}
{30,000 RMB/year, in total 120,000 RMB, CUHK-Shenzhen}{}
{}
\cventry{Sep. 2020}
{Dean's List Award}{}
{School of Science and Engineering, CUHK-Shenzhen}{}
{}
\cventry{Sep. 2021}
{Dean's List Award}{}
{School of Life and Health Sciences, CUHK-Shenzhen}{}
{}
\cventry{Sep. 2022}
{Dean's List Award}{}
{School of Life and Health Sciences, CUHK-Shenzhen}{}
{}
\cventry{Sep. 2021}
{The Second prize}{}
{Contemporary Undergraduate Mathematical Contest in Modeling, provincial level}{}
{}

% \cvsection{Courses Taken}
% \cvline{Math \& stat}{
%     Calculus I \& II, introduction to real analysis, numerical analysis,
%     ordinary differential equations, linear algebra, advanced linear algebra,
%     probability and statistics I
% }
% \cvline{Chem \& physics}{
%     Mechanics, general chemistry, organic chemistry, physical chemistry I,
%     computational (structural) biology, computational biology laboratory, biophysics,
%     molecular simulation \& modeling I
%     (including statistical mechanics theories, monte carlo, MD simulation algorithms)
% }
% \cvline{Informatics}{
%     Introduction to computer science: programming methodology,
%     computational laboratory,
%     bioinformacis,
%     computational genomics and proteomics,
%     machine learning in computational biology,
%     design and analysis of bioinformacis algorithms
% }
% \cvline{Biology}{
%     General biology, biochemistry, cell and molecular biology, genetics
% }

\cvsection{Research Experiences}
\cventry{Apr. 2021--Present}
    {Research assistant}{}
    {Hajime Hirao's group}{CUHK-Shenzhen}
    {}
\cvline{Apr. 2021--June 2021}{\small
    \textbf{Training:} theoretical studying of quantum chemistry by \textit{Modern Quantum Chemistry}
    \begin{itemize}[itemsep=2pt]
        \item SCF algorithm coding by Fortran, including RHF 6-31G \cee{H2} molecule
        and UHF 6-31G \cee{H_2^-} molecule
        \item \ul{Fixed problematic DIIS algorithm} in original group Fortran code which used for acceleration
    \end{itemize}
}
% \vspace{-1em}
% \cvline{June 2021--July 2021}{\small
%     \textbf{Training:} reaction pathway analysis--hydroxylation reaction between P450 Cpd I and propane
%     \begin{itemize}[itemsep=0pt]
%         \item Scan along the reaction path
%         \item Geometry optimization of intermediates
%         \item Calculation under different spin states
%         \item Writing scripts to extract information and generate report efficiently
%     \end{itemize}
% }
\vspace{-1em}
\cvline{Aug. 2021--Dec. 2021}{\small
    \textbf{Project:} reaction pathway analysis--P450 C-S bond formation by TleB (PDB ID: 6J83)
    \begin{itemize}[itemsep=4pt]
        % \setlength\itemsep{0.25em}
        % \item \ul{Design the whole research plan}
        \item Build truncated model to perform DFT calculations along the proposed reaction pathway to
        identify electronic configurations under different spin states
            % \item Swap electrons in $\alpha$ or $\beta$ orbitals to get stable configuration
        \item Molecular dynamics simulation of initial reaction complex
        to determine the starting path of the reaction
        \begin{itemize}
            \item Deriving MM parameters, setup system, perform MD simulations, check non-bonding interactions, check clusters in trajectory by
            statistical algorithms, found minor sub-states by clustering algorithm
            \item MMPBSA free energy approximation to compare population between states,
            in order to find which binding pose is more favorable for protein
        \end{itemize}
        \item Using quantum mechanics + molecular mechanics (QM/MM) hybrid method to
        investigate into the protein-substrate interaction
        \begin{itemize}
            \item Determine QM region of the system, use MM parameters to build up QM/MM model
            \item Use small basis set when performing optimization, then use large basis set and
            electronic embedding scheme to investigate electronic configurations and effect of
            protein
        \end{itemize}
    \end{itemize}
}
\vspace{-1em}
\cvline{Apr. 2022--Present}{\small
    \textbf{Project:} energy decomposition analysis (EDA) and natural bonding orbital (NBO) analysis of
    the nature of protein-drug interaction at the heme iron center in cytochrome P450 inhibition
    \begin{itemize}[itemsep=2pt]
        \item Write an example Lewis configuration for NBO input
        \item Performed batch EDA analysis using Q-Chem, fix convergence problem by shutdown DIIS when error is small
        \item The research could provide insight into inhibition drug design
        for P450
        \item Under review
    \end{itemize}
}

\cventry{Jan. 2020--Dec. 2020}
{Research assistant}{}
{Hsien-da Huang's group}{CUHK-Shenzhen}
{}
\cvline{}{\small
    \textbf{Project:} effects of traditional Chinese medicine in gene regulation:
    identify DEGs using statistical methods
    \begin{itemize}[itemsep=4pt]
        \item Visualization of gene expression profile using PCA and t-SNE
        to get a first sight of data's distribution
        \item Group tutorial about how to use Connectivity Map
        \begin{itemize}
            \item Exploring databases, submitting a query, interpreting statistics and heatmap
        \end{itemize}
        \item Gene set enrichment analysis (GSEA) for traditional Chinese medicines perturbed
        gene expression profile to identify differentially expressed gene sets
    \end{itemize}
}


% \cvsection{Research Interests}
% \cvline{}{Computational chemistry/biology,
% development of free-energy calculation and TS-searching algorithms,
% statistical learning applied to computational chemistry/biology.}
\vspace{-1em}
\cvsection{Language Skills}
\cvline{}{Chinese: native\newline
    English: TOEFL 107/120, with reading 30, listing 29, speaking 23, writing 25\newline
    Japanese: elementary, only able to read
}


% \section{Hobby}
% \cvline{hobby 1}{\small Description}


\end{document}


