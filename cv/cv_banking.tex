\documentclass[11pt]{moderncv}
\usepackage[scale=0.92,top=1cm,bottom=0.5cm]{geometry}
\moderncvstyle{banking}
%\usepackage{etoolbox} % already loaded by moderncv!

% bibtex reference numbering
\usepackage[style=mla,backend=bibtex,showmedium=false,maxbibnames=20]{biblatex}
\addbibresource{references.bib}
% \makeatletter
% \renewcommand*{\bibliographyitemlabel}{\@biblabel{\arabic{enumiv}}}
% \makeatother

% font settings
\usepackage{fontspec}
\setmainfont{Linux Libertine}

% chinese font
\usepackage{xeCJK}

\makeatletter
\patchcmd{\makehead}{%search
  \flushmakeheaddetails\@firstmakeheaddetailselementtrue\\\null}{%replace
  \flushmakeheaddetails\@firstmakeheaddetailselementtrue\par\vspace{-\baselineskip}\null}{%success
  }{%failure
  }
\makeatother

% personal info
\name{Haoran}{Sun}
\address{2001 Longxiang Road}{Shenzhen}{China} % Your current address
\phone[mobile]{+86 139 1029 0104}    % Your mobile phone number
\email{haoransun@link.cuhk.edu.cn} 
% \social[github]{haoran0115}
\homepage{haoran0115.github.io}

% make content compact
\linespread{0.9}

\begin{document}

\maketitle

% \cventry{year--year}{Degree}{Institution}{City}{\textit{Grade}}{Description}


%--------------------------Education----------------------------
\vspace{-3em}
\section{Education}
% cuhk
\cventry{Sept. 2019--Present}
{B.Sc., Bioinformacis; cumulative GPA: 3.716/4.000, rank 1/38; major GPA: 3.831/4.000, rank 1/38}
{Chinese University of Hong Kong, Shenzhen (CUHK-Shenzhen)}
{Shenzhen, China}
{}
{}

% ucb
\cventry{June 2022--Aug. 2022}
{Summer visiting program; GPA: 4.000/4.000}
{University of California, Berkeley}
{Berkeley, CA}
{}
{\textbf{Courses:} MATH104 Introduction to Real Analysis, MATH128A Numerical Analysis, CS61C Machine Structure.}


%---------------------------------Research------------------------------
\vspace{-0.75em}
\section{Research Experiences}
\cventry{Apr. 2022--Aug. 2022}
{Research assistant}
{Prof. Hajime Hirao's group, CUHK-Shenzhen}
{Shenzhen, China}
{}
{\textbf{Project:} Study the Bonding Nature of Fe-CO Complexes in heme Proteins, \textbf{accepted}
    \begin{itemize}\setlength\itemsep{-1pt}
        \item Wrote an example Lewis configuration of P450 Cpd I for natural bonding orbital (NBO) input.
        % \item Performed batch energy decomposition analysis (EDA) using Q-Chem. 
        \item Fixed Q-Chem SCF convergence problems by disabling DIIS algorithm when the error is small.
    \end{itemize}
}

\cventry{Aug. 2021--Dec. 2021}
{Research assistant}
{Prof. Hajime Hirao's group, CUHK-Shenzhen}
{Shenzhen, China}
{}
{\textbf {Project:} Reaction Pathway Analysis of P450 C-S Bond Formation by TleB (PDB ID: 6J83)
    \begin{itemize}\setlength\itemsep{-1pt}
    \item Built a truncated model and performed DFT calculations along the proposed reaction pathway to identify electronic configurations under different spin states.
    \item Performed molecular dynamics simulation of the initial reaction complex to determine the preferable starting structure of the reaction.
    \item Utilized quantum mechanics and molecular mechanics (QM/MM) hybrid method to investigate the protein-substrate interaction, revealing an electron transfer pattern of the initial reaction complex.
\end{itemize}
}

\cventry{Apr. 2021--June 2021}
{Research internship}
{Prof. Hajime Hirao's group, CUHK-Shenzhen}
{Shenzhen, China}
{}
{\textbf{Training:} Theoretical Studying of Quantum Chemistry by Modern Quantum Chemistry
    \begin{itemize}\setlength\itemsep{-1pt}
    \item Implemented SCF algorithm for RHF 6-31G ${H_2}$ and UHF 6-31G ${H_2^-}$ by Fortran.
    \item Fixed problems in the original DIIS algorithm, which are used for accelerating SCF algorithm .
\end{itemize}
}

\cventry{Sept. 2020--Dec. 2020}
{Research assistant}
{Prof. Hsien-da Huang's group, CUHK-Shenzhen}
{Shenzhen, China}
{}
{\textbf {Project:} Effects of Traditional Chinese Medicine on Gene Regulation
    \begin{itemize}\setlength\itemsep{-1pt}
    \item Utilized PCA and t-SNE for dimensionality reduction of gene expression profile.
    \item Arranged a group tutorial about using Connectivity Map to identify differentially expressed genes (DEGs) perturbed by traditional medicines and interpreted statistics. 
    % \item Performed gene set enrichment analysis (GSEA).
    \end{itemize}
}

%-------------------------------------Publication----------------------------------------
\vspace{-0.75em}
% \section{Publications}
\nocite{*}
\printbibliography[title={Publications}]
\vspace{-0.75em}

%------------------------------------TA----------------------------------------
\vspace{-0.75em}
\section{Teaching Experiences}
% BIM2005
\cventry{Sept. 2021--Dec. 2021}
{Undergraduate student teaching fellow, BIM2005 Computational Biology}
{CUHK-Shenzhen}
{Shenzhen, China}
{}
{\textbf{Tutorials:} docking tool Autodock-Vina; 
Hatree determinants; 
mathematical background and Numpy implementation of PCA algorithm.}

% BIM3013
\cventry{Jan. 2022--May 2022}
{Undergraduate student teaching fellow, BIM3013 Organic Chemistry}
{CUHK-Shenzhen}
{Shenzhen, China}
{}
{\textbf{Tutorials:} basic concepts of stereochemistry; mechanisms of condensation reactions.}


%-------------------------------Awards-------------------------------------
\vspace{-0.75em}
\newcommand*\dateright[1]{\hspace*{0em plus 1fill}\makebox{\textit{#1}}}
\section{Honors and Awards}
\begin{itemize}
    \item \textbf{\textit{Bowen Scholarship}}, 30,000 RMB/year, in total 120,000 RMB, CUHK-Shenzhen. \dateright{Sept. 2019--June 2023}
    \item \textbf{\textit{Dean's List Award}}, CUHK-Shenzhen. \dateright{Sept. 2020--Sept. 2022}
    \item \textbf{\textit{Contemporary Undergraduate Mathematical Contest in Modeling}}, The Second prize. \dateright{Sept. 2021}
    \item \textbf{\textit{Chinese Chemistry Olympiad}}, The First prize. \dateright{Sept. 2018}
\end{itemize}
% \cvline{}{\textbf{\textit{Bowen Scholarship}}, 30,000 RMB/year, in total 120,000 RMB, CUHK-Shenzhen \dateright{Sept. 2019--June 2023}}
% \cvline{}{\textbf{\textit{Dean's List Award}}, CUHK-Shenzhen \dateright{Sept. 2020--Sept 2022}}
% \cvline{}{\textbf{\textit{Contemporary Undergraduate Mathematical Contest in Modeling}}, The Second prize \dateright{Sept. 2021}}
% \cvline{}{\textbf{\textit{Chinese Chemistry Olympiad}}, The First prize \dateright{Sept. 2018}}

%-------------------------------Skills-------------------------------------
\vspace{-0.75em}
\section{Skills}
\begin{itemize}
    \item Coding languages: Python, Fortran, C, C++, CUDA C++ and CUDA Fortran, OpenMP, MPI, MATLAB, \LaTeX
    \item Tools: Linux (system configuration, multi-user management, software installation), WSL, Git
    \item Compt. bio./chem. tools: Amber, Gromacs, Q-Chem, Gaussian, VMD, Autodock Tools
\end{itemize}



% \cventry{}
%         {}
%         {}
%         {}
%         {}
%         {}


\end{document}

